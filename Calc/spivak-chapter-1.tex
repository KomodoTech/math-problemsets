%=========SETUP===========================
\documentclass{article}

\usepackage{fancyhdr}
\usepackage{extramarks}
\usepackage{amsmath}
\usepackage{amsthm}
\usepackage{amsfonts}
\usepackage{tikz}
\usepackage{datetime2}
\usepackage{enumerate}

\usepackage{syntonly}
%\syntaxonly

%============BASIC SETTINGS==================
\topmargin=-0.45in
\evensidemargin=0in
\oddsidemargin=0in
\textwidth=6.5in
\textheight=9.0in
\headsep=0.25in

\linespread{1,1}

\pagestyle{fancy}
\lhead{\hmwkAuthorName}
\chead{\hmwkClass\ : \hmwkTitle}
\rhead{\firstxmark}
\lfoot{\lastxmark}
\cfoot{\thepage}

\renewcommand\headrulewidth{0.4pt}
\renewcommand\footrulewidth{0.4pt}

\setlength\parindent{0pt}
\setlength\parskip{\baselineskip}

%=================CREATE PROBLEM SECTIONS====================
\newcommand{\enterProblemHeader}[1]{
    \nobreak\extramarks{}{Problem \arabic{#1} continued on next page\ldots}\nobreak{}
    \nobreak\extramarks{Problem \arabic{#1} (continued)}{Problem \arabic{#1} continued on next page\ldots}\nobreak{}
}

\newcommand{\exitProblemHeader}[1]{
    \nobreak\extramarks{Problem \arabic{#1} (continued)}{Problem \arabic{#1} continued on next page\ldots}\nobreak{}
    \stepcounter{#1}
    \nobreak\extramarks{Problem \arabic{#1}}{}\nobreak{}
}

\setcounter{secnumdepth}{0}
\newcounter{partCounter}
\newcounter{homeworkProblemCounter}
\setcounter{homeworkProblemCounter}{1}
\nobreak\extramarks{Problem \arabic{homeworkProblemCounter}}{}\nobreak{}

%
%===HOMEWORK PROBLEM ENVIRONMENT
%
% This environment takes an optional argument. When given, it will adjust the
% problem counter. This is useful for when the problems given for your
%

\newenvironment{homeworkProblem}[1][-1]{
    \ifnum#1>0
        \setcounter{homeworkProblemCounter}{#1}
    \fi
    \section{Problem \arabic{homeworkProblemCounter}}
    \setcounter{partCounter}{1}
    \enterProblemHeader{homeworkProblemCounter}
}{
    \exitProblemHeader{homeworkProblemCounter}
}

%
%===========HOMEWORK DETAILS=================
%   - Title
%   - Class
%   - Author
%
\newcommand{\hmwkTitle}{\#1. Basic Properties of Numbers}
\newcommand{\hmwkClass}{Calculus}
\newcommand{\hmwkAuthorName}{\textbf{Alexandre Leibler}}

%
%=============TITLE PAGE=========================
%

\title{
    \vspace{2in}
    \textmd{\textbf{\hmwkClass:\ \hmwkTitle}}\\
    \normalsize\vspace{0.1in}\small{Last modified\ on\ \today}\\
    \vspace{3in}
}

\author{\hmwkAuthorName}
\date{}

\renewcommand{\part}[1]{\textbf{\large Part \Alph{partCounter}}\stepcounter{partCounter}\\}

% Alias for the Solution section header
\newcommand{\solution}{\textbf{\large Solution}}


\begin{document}

    \maketitle

    \pagebreak

    \begin{homeworkProblem}
        Prove the following:
        \begin{enumerate}[(i).]
            \item If \(ax = a\) for some number \(a \neq 0\), then \(x = 1\).
            \item \((x^2 - y^2) = (x - y)(x + y)\).
            \item If \(x^2 = y^2\), then \(x = y\) or \(x = -y\).
            \item \((x^3 - y^3) = (x - y)(x^2 + xy + y^2)\).
            \item \((x^n - y^n) = (x - y)(x^{n-1} + x^{n-2}y + \cdots + xy^{n-2} + y^{n-1})\).
            \item \((x^3 + y^3) = (x + y)(x^2 - xy + y^2)\).
        \end{enumerate}
    \end{homeworkProblem}

    \begin{homeworkProblem}
        What is wrong with the following ``proof''? Let $x = y$. Then
        \begin{align*}
            x^2& = xy \text{,} \\
            x^2 - y^2& = xy - y^2 \text{,} \\
            (x + y)(x - y)& = y(x - y) \text{,}\\
            x + y& = y \text{,}\\
            2y& = y \text{,}\\
            2& = 1 \text{.}
        \end{align*}
    \end{homeworkProblem}

    \begin{homeworkProblem}
        \begin{enumerate}[(i).]
            \item \(\dfrac{a}{b} = \dfrac{ac}{bc}\), if \(b,c \neq 0\).
            \item \(\dfrac{a}{b} + \dfrac{c}{d} = \dfrac{ad + bc}{bd}\), if \(b,d \neq 0\).
            \item \((ab)^{-1} = a^{-1}b^{-1}\), if \(a,b \neq 0\).
            \item \(\dfrac{a}{b} \cdot \dfrac{c}{d} = \dfrac{ac}{db}\), if \(b,d \neq 0\).
            \item \(\left. \dfrac{a}{b} \middle/ \dfrac{c}{d} \right.\), if \(b,c,d \neq 0\).
            \item If \(b,d \neq 0\), then \(\dfrac{a}{b} = \dfrac{c}{d}\) if and only if \(ad = bc\). 
                Also determine when \(\dfrac{a}{b} = \dfrac{b}{a}\). 
        \end{enumerate}
    \end{homeworkProblem}
  
    \begin{homeworkProblem}
        Find all numbers $x$ for which
        \begin{enumerate}[(i).]
            \item \(4 - x < 3 -2x\).
            \item \(5 - x^{2} < 8\).
            \item \(5 - x^{2} < -2\).
            \item \((x - 1)(x - 3) > 0\).
            \item \(x^{2} - 2x + 2 > 0\).
            \item \(x^{2} + x + 1 > 2\).
            \item \(x^{2} - x + 10 > 16\).
            \item \(x^{2} + x + 1 > 0\).
            \item \((x - \pi)(x + 5)(x - 3) > 0\).
            \item \((x - \sqrt[3]{2})(x - \sqrt{2}) > 0\).
            \item \(2^{x} < 8\).
            \item \(x + 3^{x} < 4\).
            \item \(\dfrac{1}{x} + \dfrac{1}{1 - x} > 0\).
            \item \(\dfrac{x - 1}{x + 1} > 0\).
        \end{enumerate}
    \end{homeworkProblem}

\end{document}
